% !TeX program = pdfLaTeX-dev
\documentclass{article}

\usepackage[UKenglish]{babel}
\usepackage[T1]{fontenc}
\usepackage{microtype}

\usepackage{csquotes}

\usepackage{hyperref}
\hypersetup{hidelinks,unicode}

\usepackage{xparse}
\ExplSyntaxOn
\NewDocumentCommand\acro{m}{\textsc{\tl_lower_case:n{#1}}}
\ExplSyntaxOff
\NewDocumentCommand\email{m}{\href{mailto:#1}{#1}}
\NewDocumentCommand\foreign{m}{\textit{#1}}

% Meta-data
\author{Joseph Wright\thanks{\email{joseph.wright@morningstar2.co.uk}}}
\date{December 2019}
\title{The \url{learn-latex.org} project}

\begin{document}

\maketitle

\section{Introduction}

There are a plethora of resources available to new \LaTeX{} users. However,
determining which of these they \emph{should} use is much more difficult.
Online resources vary in quality and correctness: there are for example
well-established concerns about the quality of advice in the
\href{https://en.wikibooks.org/wiki/LaTeX}{\LaTeX{} Wikibook}. Many good
resources are also likely over-detailed for a new user: what they need is to
get over the initial barrier to becoming users.

Many programming languages are now available to try online using
web-based compilers. These are harnessed by a range of teaching websites to
offer simple introductions: a good example is \url{https://www.learnpython.org}.
Such sites tend to be strictly limited in scope, not aiming to teach every
possible idea but rather a strictly-limited beginners \enquote{menu}.

Here, I propose a new teaching site for \LaTeX{}, provisionally
called \url{learn-latex.org}. The aim of this new site would be to provide
a carefully-curated set of resources for beginners, with access to online
demonstrations directly from the \enquote{lessons}. The scope of the lessons
would be limited, and the aim of the site would be to offer the material
in \enquote{bite-sized} chunks.

\section{Content}

The UK \TeX{} Users' Group have prepared a set of notes for a one-day course
(\url{https://github.com/uktug/latex-beginners-course}): these notes were
originally prepared by Nicola Talbot and have been used successfully many
times. A similar scope of teaching is encompassed in the Overleaf
\href{https://www.overleaf.com/learn/latex/Learn_LaTeX_in_30_minutes}{\enquote{Learn
\LaTeX{} in 30 minutes}} page.

The content of the new site would draw on these resources (with appropriate
permissions), providing a set of around a dozen \enquote{lessons}, each
focussed on a basic topic. Most, if not all, of the material presented would
make use only of \LaTeX{} Team packages: commands taken from the kernel and
those from the \texttt{required} set.

In addition to the lessons, there will be a small set of resources for going
further, along with links to finding help. Much of this is already listed on
for example the \LaTeX{} team site, but having the new site self-contained is a
key aim.

\section{Design}

The appearance of the site is likely to require some input from experts in web
design. However, one may envisage a simple layout with the \enquote{lessons}
listed on one side of the screen, with the bulk of the page taken up by the
lesson. Some visual links to the design of the \LaTeX{} Team website would
likely be appropriate.

Important in the overall layout is making the site visually attractive and
accessible. This will likely differentiate it from many other \LaTeX{}
resources, many of which are several years old and reflect older website
practice.

\section{Online compilation}

A key aim of the new site will be to make learning \LaTeX{} a barrier-free
experience. Whilst having a local installation remains useful for many users,
the availability of online compilation is a particular bonus for new users.

There are a number of commercial sites offering \LaTeX{} either as an explicit
service or \enquote{behind the scenes}. Of the current set of websites,
Overleaf is perhaps the most widely-used and clearest in providing explicit
\LaTeX{} editing online. It will be important to be clear to users of the new
site that there are multiple options, but establishing a relationship with an
online provider will be important in making the site a success.

\section{Organisation}

By far the easiest way of organising a relatively small site, as envisaged
here, is to use a service such as GitHub Pages to serve the content. This will
allow a team of maintainers to create the site using simple Markdown and
\acro{html} files, and to leave the details of the web presence largely to
others.

This model has recently been adopted for the long-standing \TeX{} Frequently
Asked Questions, now served as \url{www.texfaq.org} as a custom \acro{url}
around a GitHub site. This approach is scalable and flexible, and also means
that the maintainers can be managed simply by adding/removing access to the
host repository.

Strong leadership will be important for a \enquote{canonical} teaching site for
\LaTeX{}. Whilst both \acro{tug} and \acro{dante} could act as central
points, it seems most sensible for the \LaTeX{} Team to have overall control of
a resource for teaching their core product. There are a number of individuals
outside of the team who might reasonably act as outside collaborators, for
example Nicola Talbot, Karl Berry and Barbara Beeton.

Links to any commercial collaborators will need to be carefully managed. There
are good reasons to think that online service providers will be keen on better
resources for new users. At the same time, it is vital that overall control
both of the concept and the content remains in community hands. As such,
material donated by any commercial entities will need clearly to be given
without restrictions.

\section{Funding}

Resources for hosting the new site will be modest: using free services such
as GitHub Pages and free online compilation sites should be possible. The
websites \url{learn-latex.org} and \url{learnlatex.org} have already been
secured, and are renewable at a cost of around \pounds 20 each \foreign{per
annum}.

The biggest single cost is likely to be design work at the outset. Funding
to support this effort will be required: expertise in web design is not
necessarily available readily within the \LaTeX{} community. User groups
are probably the best source for such funds, which are likely to be of
the order of \pounds 1000 to \pounds 5000.

Running costs of the proposed site will be minimal, likely limited
to \acro{url} provision.

\section{Ongoing support}

There will be need to curate the new site over time. Merging corrections or new
content will be in the control of the \enquote{curators} (likely the \LaTeX{}
Team). However, using the collaborative model offered by GitHub (or similar
hosting sites), contributions from other experienced \LaTeX{} users will be
possible. This should allow a key balance of controlled high-quality content
\emph{and} sufficient resources to maintain the site.

\section{Conclusions}

The creation of a new \LaTeX{} teaching site focussed strictly on new users and
providing a modern web experience will significantly-enhance the accessibly of
\LaTeX{} by new users. The effort required to establish and maintain this site
is reasonable. In particular, the more web-focussed aspects can likely be
handled on a contract basis. This will allow the \LaTeX{} experts involved to
focus on content creation and review. The latter can be handled using standard
Git workflows to allow third-party contributions
to the site without endangering overall quality.

Funding at the level of \pounds 1000 to \pounds 5000 at outset will be a
one-off cost, after which the site will be broadly self-sustaining in
financial terms.

\end{document}
