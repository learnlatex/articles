\documentclass[harvardcite]{ltugboat}

\usepackage[UKenglish]{babel}
\usepackage{csquotes}
\usepackage{hologo}
\usepackage{xurl}
\usepackage{hyperref}

%%%%%%%%%%%%%%%%%%%%%%%%%%%%%%%%%%%%%%%%%%%%%%%%%%%%%%%%%%%%%%%%%%%%%
% Some (very simple) new commands are defined.
%%%%%%%%%%%%%%%%%%%%%%%%%%%%%%%%%%%%%%%%%%%%%%%%%%%%%%%%%%%%%%%%%%%%%
\usepackage{xparse}

%%%%%%%%%%%%%%%%%%%%%%%%%%%%%%%%%%%%%%%%%%%%%%%%%%%%%%%%%%%%%%%%%%%%%
% Meta-data for this paper
%%%%%%%%%%%%%%%%%%%%%%%%%%%%%%%%%%%%%%%%%%%%%%%%%%%%%%%%%%%%%%%%%%%%%
\title{\texttt{learnlatex.org}: Taking \LaTeX{} training fully interactive}
\author{Joseph Wright}
\address{Northampton, United Kingdom}
\netaddress{joseph dot wright (at) morningstar2.co.uk}

\begin{document}

\maketitle

\section{Introduction}

There are a plethora of resources available to new \LaTeX{} users. However, it
is much more difficult to discover which of these provides the best
introduction to \LaTeX{}. These online resources vary in quality and correctness:
over time, and with limited editing, even good advice can become out-of-date.
Many good resources are over-detailed for a new user who needs only
straightforward help to get over the initial barrier to using the system.

For many programming languages there are now web-sites providing the
opportunity to try coding online using a cloud compiler. These cloud compilers
can be harnessed by a range of teaching websites to offer a simple introduction
to the language using a suitable \acro{IDE} (integrated development
environment): a good example is \url{LearnPython.org}. Such sites tend to be
limited in scope as they are not aiming to teach every possible idea in the
language but only \enquote{Beginners} menu.

Over the past six months, work has been ongoing in filling that gap for
\LaTeX{}: \url{learnlatex.org}. The aim of this new site is to provide a
carefully-curated set of resources for beginner \LaTeX{} authors, with integrated
use of an online \LaTeX{} environment and demonstrations accessed directly from
these lessons. The scope of these learning resources is focussed, and with the
aim of offering the material in bite-sized chunks.

\section{Existing resources}

There are already many online resources for learning \LaTeX{}, as a simple web
search will reveal. The top hit at present is Overleaf's \enquote{Learn
\LaTeX{} in 30 Minutes}
(\url{https://www.overleaf.com/learn/latex/Learn_LaTeX_in_30_minutes}), which
covers a lot of the core ideas with a series of examples. Second is
\url{https://www.latex-tutorial.com/}, which takes the step of dividing up the
lessons into a series of pages. There are then more \enquote{traditional}
resources, including the long-standing \LaTeX{} Wikibook
(\url{https://en.wikibooks.org/wiki/LaTeX}), and of course many excellent
printed \LaTeX{} guides.

The Overleaf page is notable as Overleaf provides a full \LaTeX{} system
online, and is used by many people as their \LaTeX{} editor/system. On their
teaching page, the examples are given as code blocks but to run then, a
separate tab needs to be opened. That has the positive of looking exactly the
same as any other document edited on Overleaf, but means leaving the teaching
page. The other leading hits for learning \LaTeX{} require that the learner
copies or types out the examples by-hand.

In contrast, learning other programming languages can today often start on
websites that let them try out the system \emph{in the page}. That means
no copy-pasting, no need to move to other tabs and no need to download and
install software. As mentioned above, \url{LearnPython.org} is an excellent
example, forming part of a family of around half a dozen sites using the
same overall framework to teach a set of modern languages.

The \url{learnlatex.org} project was started with the aim of combining existing
best-practice teaching on \LaTeX{} with a strong focus on interactive, online,
training. That required tackling three things: the content, the website
structure and the online element.

\section{Writing content}

\section{Website structure}

\section{Online \LaTeX{} compilers}

\section{How you can help}

\makesignature

\end{document}
