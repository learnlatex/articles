\documentclass[harvardcite]{ltugboat}

\usepackage[UKenglish]{babel}
\usepackage{csquotes}
\usepackage{hologo}
\usepackage{xurl}
\usepackage{hyperref}

%%%%%%%%%%%%%%%%%%%%%%%%%%%%%%%%%%%%%%%%%%%%%%%%%%%%%%%%%%%%%%%%%%%%%
% Some (very simple) new commands are defined.
%%%%%%%%%%%%%%%%%%%%%%%%%%%%%%%%%%%%%%%%%%%%%%%%%%%%%%%%%%%%%%%%%%%%%
\usepackage{xparse}

%%%%%%%%%%%%%%%%%%%%%%%%%%%%%%%%%%%%%%%%%%%%%%%%%%%%%%%%%%%%%%%%%%%%%
% Meta-data for this paper
%%%%%%%%%%%%%%%%%%%%%%%%%%%%%%%%%%%%%%%%%%%%%%%%%%%%%%%%%%%%%%%%%%%%%
\title{\texttt{learnlatex.org}: Taking \LaTeX{} training fully interactive}
\author{Joseph Wright}
\address{Northampton, United Kingdom}
\netaddress{joseph dot wright (at) morningstar2.co.uk}

\begin{document}

\maketitle

\section{Introduction}

There are a plethora of resources available to new \LaTeX{} users. However, it
is much more difficult to discover which of these provides the best
introduction to \LaTeX{}. These online resources vary in quality and correctness:
over time, and with limited editing, even good advice can become out-of-date.
Many good resources are over-detailed for a new user who needs only
straightforward help to get over the initial barrier to using the system.

For many programming languages there are now web-sites providing the
opportunity to try coding online using a cloud compiler. These cloud compilers
can be harnessed by a range of teaching websites to offer a simple introduction
to the language using a suitable \acro{IDE} (integrated development
environment): a good example is \url{LearnPython.org}. Such sites tend to be
limited in scope as they are not aiming to teach every possible idea in the
language but only \enquote{Beginners} menu.

Over the past six months, work has been ongoing in filling that gap for
\LaTeX{}: \url{learnlatex.org}. The aim of this new site is to provide a
carefully-curated set of resources for beginner \LaTeX{} authors, with integrated
use of an online \LaTeX{} environment and demonstrations accessed directly from
these lessons. The scope of these learning resources is focussed, and with the
aim of offering the material in bite-sized chunks.

\section{Existing resources}

There are already many online resources for learning \LaTeX{}, as a simple web
search will reveal. The top hit at present is Overleaf's \enquote{Learn
\LaTeX{} in 30 Minutes}
(\url{https://www.overleaf.com/learn/latex/Learn_LaTeX_in_30_minutes}), which
covers a lot of the core ideas with a series of examples. Second is
\url{https://www.latex-tutorial.com/}, which takes the step of dividing up the
lessons into a series of pages. There are then more \enquote{traditional}
resources, including the long-standing \LaTeX{} Wikibook
(\url{https://en.wikibooks.org/wiki/LaTeX}), and of course many excellent
printed \LaTeX{} guides.

The Overleaf page is notable as Overleaf provides a full \LaTeX{} system
online, and is used by many people as their \LaTeX{} editor/system. On their
teaching page, the examples are given as code blocks but to run then, a
separate tab needs to be opened. That has the positive of looking exactly the
same as any other document edited on Overleaf, but means leaving the teaching
page. The other leading hits for learning \LaTeX{} require that the learner
copies or types out the examples by-hand.

In contrast, learning other programming languages can today often start on
websites that let them try out the system \emph{in the page}. That means
no copy-pasting, no need to move to other tabs and no need to download and
install software. As mentioned above, \url{LearnPython.org} is an excellent
example, forming part of a family of around half a dozen sites using the
same overall framework to teach a set of modern languages.

The \url{learnlatex.org} project was started with the aim of combining existing
best-practice teaching on \LaTeX{} with a strong focus on interactive, online,
training. That required tackling three things: the content, the website
structure and the online element.

\section{Writing content}

The starting point for writing the content was long-standing material developed
by the Nicola Talbot then used by the UK \TeX{} User Group (\acro{uk-tug}) for
face-to-face lessons (https://github.com/uktug/latex-beginners-course). After 
assembling a small group of volunteers, we began by looking at this curriculum,
the content of other sites and discussions we have all had with new(er) users.
That lead to a first set of section titles, which we adjusted as new ideas
arose.

We then started drafting the content, taking existing notes plus new material
to deliver around 15 lessons.\footnote{We've grown to 16 lessons, as there was
a strong argument to add one on errors.} As we worked through these, we found
we had more to say that was reasonable to cover in a focussed set of lessons.
We also felt there was good content that did not belong in the basic lessons,
but did belong somewhere. That led to a split of each lesson into two parts:
the basics and \enquote{going further}. (We'll look at the file structures
below, and how they build the site.)

As well as the content and demonstrations, we've worked hard to make sure that
the lessons are up-to-date and accurate. That's led to discussions with the
\LaTeX{} team about for example \verb|\label|, which is reflected in the lesson
content and means that we are confident the site covers \emph{best practice}.

\section{Website structure}

Based on experience with \url{texfaq.org}, we knew that GitHub Pages, which
lets you write webpages in Markdown, was a good place to look to host the site.
This means that multiple authors can easily work with the sources without
needing to set up complex build system: the pages can be edited on the GitHub
site, so a local Git installation is not even required.

We started with a simple structure, with an index and 15 \texttt{lesson-XX}
files (all in Markdown, of course). GitHub Pages runs a system called Jekyll,
which turns those files into the full site. That's done using various bits of
template plus some scripting, which can all be user-controlled. As the site has
grown, we split the extra content into \texttt{more-XX} files, which can then
be linked automatically to their \enquote{parent}.

Not long after we started work, the question of translations came up. We'd started
with the idea of just working in English, but we didn't want to box ourselves in.
So we moved all of the content into a directory called \texttt{/en}, and started
exploring how to add a language selector. To help with that, we made some stub
files that were marked as being in a few languages, then got at least the
page titles translated by real people. It turns out to be reasonably easy
to add a selector, and there's already been interest in translation, so
we have the site in English, Vietnamese and Portuguese already, with Spanish
and German in the pipeline.

\section{Online \LaTeX{} compilers}

\section{What's next}

The web design we are using is very basic: just what comes out-of-the-box with
GitHub Pages. To help users navigate, we need proper development of a full
site. That's beyond the expertise of the site team, but we have been able to
raise funds to employ a professional web developer. He's looking at design both
in terms of appearance and structure: things like mobile accessibility,
metadata, etc.\ are all really important.

There's also still more to do on the content, in particular working out if we
need more formalised exercises or lesson summaries. That likely needs user
feedback, for which we need \emph{your} help. The site will work best if it's
promoted to potential users, and readers of \emph{TUGboat} are we hope
well-placed to recommend it. Of course, to do that, we'd encourage you to read
the site, give feedback and perhaps even suggest some improvements!

\makesignature

\end{document}
