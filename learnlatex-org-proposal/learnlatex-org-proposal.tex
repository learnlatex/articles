% !TeX program = pdfLaTeX-dev
\documentclass{article}

\usepackage[UKenglish]{babel}
\usepackage[T1]{fontenc}
\usepackage{microtype}

\usepackage{csquotes}

\usepackage{hyperref}
\hypersetup{hidelinks,unicode}

\usepackage{xparse}
\ExplSyntaxOn
\NewDocumentCommand\acro{m}{\textsc{\tl_lower_case:n{#1}}}
\ExplSyntaxOff
\NewDocumentCommand\email{m}{\href{mailto:#1}{#1}}
\NewDocumentCommand\foreign{m}{\textit{#1}}

% Meta-data
\author{Joseph Wright\thanks{\email{joseph.wright@morningstar2.co.uk}}}
\date{February 2020}
\title{The \url{learnlatex.org} project}

\begin{document}

\maketitle

\section{Introduction}

There are a plethora of resources available to new \LaTeX{} users. However, it
is much more difficult to discover which of these provides the best
introduction to \LaTeX{}. These online resources vary in quality and
correctness: there are for example well-established concerns about the quality
of advice in the \href{https://en.wikibooks.org/wiki/LaTeX}{\LaTeX{} Wikibook}.
Many good resources are over-detailed for a new user who needs only
straightforward help to get over the initial barrier to using the system.

For many \enquote{programming languages} there are now web-sites providing the
opportunity to try coding online using a cloud compiler. These cloud compilers
can be harnessed by a range of teaching websites to offer a simple introduction
to the language using a suitable \acro{ide} (integated development
environment): a good example is \url{https://www.learnpython.org}. Such sites
tend to be strictly limited in scope as they are not aiming to teach every
possible idea in the language but only a strictly-limited \enquote{Beginners
menu}.

Here I propose a new teaching site for \LaTeX{}, provisionally called
\url{learn-latex.org}. The aim of this new site would be to provide a
carefully-curated set of resources for beginner \LaTeX{} authors, with
integrated use of an online \LaTeX{} environment and demonstrations accessed
directly from these \enquote{lessons}. The scope of these learning resources
would be limited, and the aim of the site would be to offer the material in
\enquote{bite-sized} chunks.

\section{Content}

The UK \TeX{} Users' Group have prepared a set of notes for a one-day course
(\url{https://github.com/uktug/latex-beginners-course}): these notes were
originally prepared by Nicola Talbot and have been used successfully many
times. A similar teaching scope is that provided in the Overleaf
\href{https://www.overleaf.com/learn/latex/Learn_LaTeX_in_30_minutes}{\enquote{Learn
\LaTeX{} in 30 minutes}} page.

The content of the new site will draw on these two resources (with appropriate
permissions), providing a set of around a dozen \enquote{lessons}, each
focussed on a basic topic. Most, if not all, of the material will be taken from
the basic (kernel) ideas or, in some cases, from \LaTeX{} Team packages: only
document-level commands taken from base and the \texttt{required} set of
packages.

In addition to the learning resources, there will be a small set of suggestions
for \enquote{going further}, along with links to finding help. Much of this
extra material is already available elsewhere online via, for example, the
\LaTeX{} team site; but making this new site self-contained is a key aim.

To enable the content to be used as widely as possible, licensing is an
important consideration. At this stage, a dual license approach seems most
suitable. For the web pages themselves, the Creative Commons CC-BY-SA 4.0
license will allow re-use with attribution. For code samples, a more permissive
license is likely best: the CC0 (roughly equivalent to Public Domain but
applying in jurisdictions which do no have the PD concept) is suggested. This
will allow users to simply pick and and use the samples in their own documents
without any legal concerns.

\section{Design}

The appearance and mechanics of the site will require some input from experts
in web site design. One could, for example, envisage a simple layout with the
bookmarks listed on one side of the screen and the bulk of the page taken up by
the \enquote{learning materials}. Some visual relationship to the design of the
\LaTeX{} Team website would be appropriate.

Making the site visually attractive and, in all ways, accessible will be an
important aim as this will help to differentiate it from many other online
\LaTeX{} resources (many of which are several years old and reflect older
website practice).

At a technical level, the site will need to be fully-compliant with Web Content
Accessibility Guidelines (WCAG). These are particularly important to
visually-impaired users, for whom \LaTeX{} offers significant advantage.
Created a new site using modern frameworks should make this straight-forward,
and it will be part of the brief for the web designer contracted to carry out
the design work.

\section{Online compilation}

A key aim of the new site will be to make learning \LaTeX{} a
\enquote{barrier-free experience} for new users: therefore it needs to provide
a well-integrated online compiler.

There are a number of commercial sites offering \LaTeX{}, either as an explicit
service or \enquote{behind the scenes}. Of the current set of websites,
Overleaf is perhaps the most widely-used and it is the leader in providing
\LaTeX{} processing online. It will be important to be clear to users of the
new site that there are multiple options for \enquote{compiling documents}, but
establishing a relationship with an online provider will be important in making
the site a success.

\section{Technical Organisation}

By far the easiest way of organising a relatively small site, as envisaged
here, is to use a service such as GitHub Pages to serve the content. This will
allow a team of maintainers to create the site using simple Markdown and
\acro{html} files, leaving the details of the web presence to the service
providers.

This model has recently been adopted for the long-standing \TeX{}
\textsl{Frequently Asked Questions}, now served as \url{www.texfaq.org} as a
custom \acro{url} around a GitHub site. This approach is scalable and flexible,
and also means that the maintenance personnel can be managed simply, by
adding/removing an individual's access to the host repository.

Strong institutional support will be important for such a \enquote{canonical}
learning resources site for \LaTeX{}. Whilst both \acro{tug} and \acro{dante}
could act as central points, it seems most sensible for the \LaTeX{} Team to
have overall editorial and operational control of such resources for their core
product. There are a number of individuals outside this team who would be
useful collaborators, for example Nicola Talbot, Karl Berry and Barbara Beeton.

Relationships with commercial collaborators will need to be carefully managed.
There are good reasons to think that online service providers will be keen on
supporting such improved online learning resources for new users. At the same
time, it is vital that overall control, of both the concept and the content,
remains in the hands of the \LaTeX{} Team. As such, material donated by any
commercial entities will need clearly to be given without restrictions.

\section{Funding}

Resources for hosting the new site will be modest due to the use of many free
services such as GitHub Pages and online compilation sites. The websites
\url{learn-latex.org} and \url{learnlatex.org} have already been secured, and
are renewable at a cost of around \pounds 20 (\acro{gbp}) each a year.

The biggest single cost is likely to be for the design work on the web-site at
the outset. Funding to support this effort will be required: unpaid expertise
in web design is not available readily within the \LaTeX{} community. User
groups are probably the best source for such funds, which are likely to be of
the order of \pounds 4000 to \pounds 6000.

Running costs of the proposed site will be minimal, likely limited
to \acro{url} provision.

\section{Ongoing support}

There will be an ongoing need to curate the site: merging corrections and
adding new content will be under the control of the \enquote{curators} (likely
the \LaTeX{} Team). However, using the collaborative model offered by GitHub
(or similar hosting sites), adding contributions from other experienced
\LaTeX{} users and teachers will be possible. This should allow a good
symbiosis of controlled, high-quality content \emph{and} sufficient resources
to maintain the site.

\section{Conclusions}

The creation of a new, high-quality \LaTeX{} teaching site focussed strictly on new author users and
providing a modern web experience will significantly enhance the spread of
\LaTeX{} to new users. The effort required to establish and maintain this site
will be reasonable, with the more web-focussed aspects likely to be
handled on a contract basis. This will allow  \LaTeX{} teaching experts to
focus on content creation and review. These can be handled using standard
Git workflows to allow third-party contributions
to the site without endangering overall quality.

Funding at the level of \pounds 4000 to \pounds 6000 at outset will be a
one-off cost, after which the site will be broadly self-sustaining in
financial terms.

\end{document}



